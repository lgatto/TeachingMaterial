\documentclass{beamer}
\usetheme{CambridgeUS}
\usepackage{color}

\title[Cambridge R Course]{A beginner's guide\\to solving biological problems\\in R}
\subtitle{Day 1 Morning\\}
\author[]{Robert Stojni\'{c} (rs550) \and Laurent Gatto (lg390) \and Rob Foy (raf51) \and John Davey (jd626)}
\date[]{Course materials: \url{http://logic.sysbiol.cam.ac.uk/teaching/Rcourse/}\\Slides by Ian Roberts and Robert Stojni\'{c}}
\begin{document}
\begin{frame}
    \titlepage
\end{frame}


\begin{frame}{Day 1 Schedule}

\begin{itemize}
  \item The R environment and basics
  \begin{itemize}
      \item Where to get R
      \item Brief introduction to essential R
      \item R help, scripting and packages
  \end{itemize}
\end{itemize}
\textcolor{green}{Morning coffee}
\begin{itemize}
  \item Objects and data types
  \begin{itemize}
      \item Learn how to input and manipulate data
  \end{itemize}
\end{itemize}
\textcolor{green}{Lunch}
\begin{itemize}
    \item Introduction to essential R commands
    \begin{itemize}
        \item Base functions
        \item Read and write data
    \end{itemize}
\end{itemize}
\textcolor{green}{Afternoon coffee}
\begin{itemize}
    \item R for data analysis
    \begin{itemize}
        \item Statistical tests and maths support
    \end{itemize}
\end{itemize}

\end{frame}

\begin{frame}{What's R?}
    \begin{itemize}
        \item A statistical programming environment
        \begin{itemize}
            \item based on S
            \item Suited to high level data analysis
        \end{itemize}
        \item Open source \& cross platform
        \item Extensive graphics capabilities
        \item Diverse range of add-on packages
        \item Active community of developers
        \item Thorough documentation
    \end{itemize}
\end{frame}

\begin{frame}{The environment and basics}
\end{frame}

\begin{frame}{Screenshot}
\end{frame}


\begin{frame}{Various platforms supported}
    \begin{itemize}
        \item Release 2.15.0 (March 2012)
        \begin{itemize}
            \item Base package
            \item Contributed packages (general purpose extras)
            \item  Over 4000 packages available
        \end{itemize}
        \item Windows
        \begin{itemize}
            \item \url{http://www.stats.bris.ac.uk/R/bin/windows/base/R-2.11.1-win32.exe}
        \end{itemize}
        \item Mac OS (10.2+)
            \begin{itemize}
                \item \url{http://cran.r-project.org/bin/macosx/}
            \end{itemize}
        \item Linux
            \begin{itemize}
                \item \url{http://cran.r-project.org/bin/linux/}
            \end{itemize}
        \item Executed using command line, or a graphical user interface
        \begin{itemize}
            \item Will demonstrate both, and use all-platform GUI, RStudio
        \end{itemize}
    \end{itemize}
\end{frame}

\begin{frame}{Getting Started}
    \begin{itemize}
        \item R is a program which, once installed on your system, can be launched and is immediately ready to take input directly from the user
        \item There are two ways to launch R:
        \begin{enumerate}
            \item From the command line (particularly useful if you're quite familiar with Linux)
            \item As an application called RStudio (very good for beginners)
        \end{enumerate}
    \end{itemize}
\end{frame}

\begin{frame}[fragile]{Prepare to launch R}
    \framesubtitle{From command line}
    \begin{itemize}
        \item To start R in Linux we need to enter the Linux console (also called Linux terminal and Linux shell)
        \item To start R, at the prompt simply type:
        \begin{verbatim}
$ R\end{verbatim}
        \item If R doesn't print the welcome message, call us to help!
    \end{itemize}
\end{frame}

\begin{frame}{Prepare to launch R}
    \framesubtitle{Using RStudio}
    \begin{itemize}
        \item To launch RStudio, find the Rstudio icon in the menu bar on the left of the screen and double-click
    \end{itemize}
\end{frame}

\begin{frame}[fragile]{The Working Directory (wd)}
    \begin{itemize}
        \item Like many programs R has a concept of a working directory (wd)
        \item It is the place where R will look for files to execute and where it will save files, by default
        \item For this course we need to set the working directory to the location of the course scripts
        \item At the command prompt in the terminal or in RStudio console type:
        \begin{verbatim}
> setwd("R_course/Day_1_scripts")\end{verbatim}
        \item Alternatively in RStudio use the mouse and browse to the directory location
        \item Tools - Set Working Directory - Choose Directory...
    \end{itemize}
\end{frame}

\begin{frame}[fragile]{Basic concepts in R}
    \framesubtitle{command line calculation}
    \begin{itemize}
        \item The command line can be used as a calculator. Type:
        \begin{verbatim}
> 2 + 2
[1] 4

> 20/5 - sqrt(25) + 3^2
[1] 8

>sin(pi/2)
[1] 1\end{verbatim}
        \item Note: The number in the square brackets is an indicator of the position in the output. In this case the output is a vector of length 1 (i.e. a single number). More on vectors coming up...
    \end{itemize}
\end{frame}

\begin{frame}[fragile]{Basic concepts in R}
    \framesubtitle{variables}
    \begin{itemize}
        \item A variable is a letter or word which takes (or contains) a value. We use the assignment `operator', {\tt <-}
        \begin{verbatim}
> x <- 10
> x
[1] 10
> myNumber <- 25
> myNumber
[1] 25 \end{verbatim}
\item We can perform arithmetic on variables:
\begin{verbatim}
> sqrt(myNumber)
[1] 5\end{verbatim}
\item We can add variables together:
\begin{verbatim}
> x + myNumber
[1] 35\end{verbatim}
    \end{itemize}
\end{frame}

\begin{frame}[fragile]{Basic concepts in R}
    \framesubtitle{variables}
    \begin{itemize}
        \item We can change the value of an existing variable:
        \begin{verbatim}
> x <- 21
> x
[1] 21\end{verbatim}
        \item We can modify the contents of a variable:
\begin{verbatim}
> myNumber <- myNumber + sqrt(16)
[1] 29\end{verbatim}
    \end{itemize}
\end{frame}

\begin{frame}[fragile,shrink]{Basic concepts in R}
    \framesubtitle{functions}
    \begin{itemize}
        \item \textbf{Functions} in R perform operations on \textbf{arguments} (the input(s) to the function). We have already used \textbf{sin(x)} which returns the sine of \textbf{x}. In this case the function has one argument, \textbf{x}. Arguments are \emph{always} contained in parentheses, i.e. curved brackets \textbf{()}, separated by commas.
        \item Some other common functions: \textbf{floor()}, \textbf{sum()}, \textbf{max()}, \textbf{mean()}
        \item Try these:
        \begin{verbatim}
> floor(3.142)
[1] 3
> sum(3, 4, 5, 6)
[1] 18
> max(3, 4, 5, 6)
[1] 6
> mean(3, 4, 5, 6)
[1] 3\end{verbatim}
        \item Something has gone wrong with the last function. We need to understand more about vectors...
    \end{itemize}
\end{frame}

\begin{frame}[fragile,shrink]{Basic concepts in R}
    \framesubtitle{vectors}
    \begin{itemize}
        \item The function \textbf{c()} \emph{combines} its arguments into a \textbf{vector}
        \begin{verbatim}
> x <- c(3, 4, 5, 6)
> x
[1] 3 4 5 6\end{verbatim}
        \item As mentioned, the square brackets \textbf{[]} indicate position within the vector. We can extract individual elements by using the \textbf{[]} notation.
        \begin{verbatim}
> x[1]
[1] 3
> x[4]
[1] 6
        \end{verbatim}
        \item We can even put a vector inside the square brackets.
        \begin{verbatim}
> y <- c(2, 3)
> x[y]
[1] 4 5\end{verbatim}
        \item We can now solve the problem from the previous slide:
        \begin{verbatim}
> mean(x)
[1] 4.5\end{verbatim}
    \end{itemize}
\end{frame}


\begin{frame}[fragile,shrink]{Basic concepts in R}
    \framesubtitle{vectors}
    \begin{itemize}
        \item There are a number of shortcuts to create a vector. Instead of:
        \begin{verbatim}
> x <- c(3, 4, 5, 6, 7, 8, 9, 10, 11, 12)\end{verbatim}
        \item write
        \begin{verbatim}
> x <- 3:12\end{verbatim}
        \item Using the \textbf{seq()} function...
        \begin{verbatim}
> x <- seq(2, 10, 2)
> x
[1] 2 4 6 8 10
> x <- seq(2, 10, length.out=7)
> x
[1] 2.00000 3.33333 4.66667 6.00000 7.33333 8.66667 10.00000\end{verbatim}
        \item or the \textbf{rep()} function
        \begin{verbatim}
> y <- rep(3, 5)
> y
[1] 3 3 3 3 3
> y <- rep(1:3, 5)
> y
[1] 1 2 3 1 2 3 1 2 3 1 2 3 1 2 3\end{verbatim}
    \end{itemize}
\end{frame}

\begin{frame}[fragile,shrink]{Basic concepts in R}
    \framesubtitle{vectors}
    \begin{itemize}
        \item We have seen some ways of extracting elements of a vector. We can use these shortcuts to make things easier (or more complex!)
        \begin{verbatim}
> x <- 3:12
> x[3:7]
[1] 5 6 7 8 9
> x[seq(2, 6, 2)]
[1] 4 6 8 
> x[rep(3, 2)]
[1] 5 5\end{verbatim}
        \item We can add an element to a vector
        \begin{verbatim}
> y <- c(x, 1)
> y
[1] 3 4 5 6 7 8 9 10 11 12 1\end{verbatim}
        \item We can glue vectors together
        \begin{verbatim}
> z <- c(x, y)
> z
[1] 3 4 5 6 7 8 9 10 11 12 3 4 5 6 7 8 9 10 11 12 1\end{verbatim}
    \end{itemize}
\end{frame}

\begin{frame}[fragile,shrink]{Basic concepts in R}
    \framesubtitle{vectors}
    \begin{itemize}
        \item We can remove element(s) from a vector
        \begin{verbatim}
> x <- 3:12
> x[-3]
[1] 3 4 6 7 8 9 10 11 12
> x[-(5:7)]
[1] 3 4 5 6 10 11 12
> x[-seq(2, 6, 2)]
[1] 3 5 7 9 10 11 12\end{verbatim}
        \item Finally, we can modify the contents of a vector
        \begin{verbatim}
> x[6] <- 4
> x
[1] 3 4 5 6 7 4 9 10 11 12
> x[3:5] <- 1
> x
[1] 3 4 1 1 1 4 9 10 11 12\end{verbatim}
    \item Remember! \textbf{Square} brackets for indexing \textbf{[]}, \textbf{parentheses} for function arguments \textbf{()}.
    \end{itemize}
\end{frame}

\begin{frame}[fragile,shrink]{Basic concepts in R}
    \framesubtitle{vector arithmetic}
    \begin{itemize}
        \item When applying all standard arithmetic operations to vectors, application is element-wise.
        \begin{verbatim}
> x <- 1:10
> y <- x*2
> y
[1] 2 4 6 8 10 12 14 16 18 20
> z <- x^2
> z
[1] 1 4 9 16 25 36 49 64 81 100\end{verbatim}
        \item Adding two vectors
        \begin{verbatim}
> y + z
[1] 3 8 15 24 35 48 63 80 99 120\end{verbatim}
        \item Vectors don't have to be the same length (what's this?)...
        \begin{verbatim}
> x + 1:2
[1] 2 4 4 6 6 8 8 10 10 12\end{verbatim}
        \item but that doesn't always work:
        \begin{verbatim}
> x + 1:3 (...?)\end{verbatim}
    \end{itemize}
\end{frame}

\begin{frame}[fragile,shrink]{Writing scripts with Rstudio}
Typing lots of commands directly to R can be tedious. A better way is to write the command to a file and then load it into R.
        \begin{itemize}
            \item Click on \textbf{File - New} in RStudio
            \item Type in some R code, e.g.
        \begin{verbatim}
    x <- 2 + 2
    print(x)\end{verbatim}
        \item Click on \textbf{Run} to execute the \textbf{current line}, and \textbf{Source} to execute the \textbf{whole script}.
        \item Sourcing can also be performed manually with {\tt source("myScript.R")}
    \end{itemize}
\end{frame}

\begin{frame}[fragile,shrink]{Getting Help}
    \begin{itemize}
        \item To get help on any R function, type {\tt ?} followed by the function name. For example:
        \begin{verbatim}
> ?seq
        \end{verbatim}
        \item This retrieves the syntax and arguments for the function. It also tells you which \textbf{package} it belongs to. There will typically be example usage.
        \item If you can't remember the exact name type {\tt ??} followed by your guess. R will return a list of possibles.
        \begin{verbatim}
> ??rint
        \end{verbatim}

    \end{itemize}
\end{frame}

\begin{frame}[fragile,shrink]{Interacting with the R console}
    \begin{itemize}
        \item R console symbols
        \begin{itemize}
            \item {\tt \;} end of line
            \begin{itemize}
                \item Enables multiple commands to be placed on one line of text
            \end{itemize}
            \item {\tt \#} comment
            \begin{itemize}
                \item indicates text is a comment and not executed
            \end{itemize}
            \item {\tt +} command line wrap
            \begin{itemize}
                \item R is waiting for you to complete an expression
            \end{itemize}        \end{itemize}

        \item \textbf{Ctrl-c} or \textbf{escape} to clear input and try again
        \item \textbf{Ctrl-l} to clear window
        \item Press {\tt q} to leave help (using R from the terminal)
        \item Use the \textbf{TAB key} for command auto completion
        \item Use \textbf{up and down arrows} to scroll through the command history

    \end{itemize}
\end{frame}

\begin{frame}[fragile,shrink]{R packages}
    \begin{itemize}
        \item R comes ready loaded with various libraries of functions called \textbf{packages}, e.g. the function \textbf{sum()} is in the \textbf{base} package and \textbf{sd()}, which calculates the standard deviation of a vector, is in the \textbf{stats} package
        \item There are 1000s of additional packages provided by third parties, and the packages can be found in numerous server locations on the web called \textbf{repositories}
        \item The two repositories you will come across the most are
        \begin{itemize}
            \item \textbf{The Comprehensive R Archive Network (CRAN)}
            \item Bioconductor
        \end{itemize}
        \item CRAN is mirrored in many locations. Set your local mirror in RStudio using Tools - Options, and choose a CRAN mirror
        \item Set the Bioconductor package download tool by typing:
        \begin{verbatim}
> source ("http://bioconductor.org/biocLite.R")\end{verbatim}
        \item Bioconductor packages are then loaded with the \textbf{biocLite()} function:
        \begin{verbatim}
> biocLite("PackageName")
        \end{verbatim}
    \end{itemize}
\end{frame}

\begin{frame}[fragile,shrink]{R packages}
    \begin{itemize}
        \item 3900+ packages on CRAN:
        \begin{itemize}
            \item Use CRAN search to find functionality you need:\\
            \url{http://cran.r-project.org/search.html}
            \item Or, look for packages by theme:\\
            \url{http://cran.r-project.org/web/views/}
        \end{itemize}
        \item 550+ packages in Bioconductor:
        \begin{itemize}
            \item Specialised in genomics:\\
            \url{http://www.bioconductor.org/packages/release/bioc/}
        \end{itemize}
        \item \textbf{Other repositories:}
        \item 1000+ projects on R-forge:\\
        \url{http://r-forge.r-project.org/}
        \item R graphical manual:\\
        \url{http://bg9.imslab.co.jp/Rhelp/}
    \end{itemize}
\end{frame}

\begin{frame}[fragile,shrink]{Exercise: Install Packages Matrix and aCGH}
    \begin{itemize}
        \item Matrix is a CRAN extras package
        \begin{itemize}
            \item Use \textbf{install.packages()} function...
            \begin{verbatim}
install.packages("Matrix")\end{verbatim}
\item or in RStudio goto Tools - Install Packages... and type the package name
        \end{itemize}
        \item aCGH is a Bioconductor package (\url{www.bioconductor.org})
        \begin{itemize}
            \item Use \textbf{biocLite()} function
            \begin{verbatim}
biocLite("aCGH")\end{verbatim}
        \end{itemize}
        \item R needs to be told to use the new functions from the installed packages
        \begin{itemize}
            \item Use \textbf{library()} function to load the newly installed features
            \begin{verbatim}library("Matrix") # loads matrix functions\end{verbatim}
            \begin{verbatim}library("aCGH") # loads aCGH functions\end{verbatim}
            \item {\tt library()}
            \begin{itemize}
                \item Lists all the packages you've got installed locally
            \end{itemize}
        \end{itemize}
    \end{itemize}
\end{frame}

\begin{frame}[fragile,shrink]{MORNING COFFEE}
Morning coffee
\end{frame}

\begin{frame}[fragile,shrink]{OBJECTS}
Objects
\end{frame}



\begin{frame}[fragile,shrink]{}
    \begin{itemize}
        \item
        \begin{verbatim}
        \end{verbatim}
    \end{itemize}
\end{frame}

\end{document}