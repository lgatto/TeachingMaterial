% Created 2012-10-30 Tue 12:39
\documentclass{beamer}
\usepackage[utf8]{inputenc}
\usepackage[T1]{fontenc}
\usepackage{fixltx2e}
\usepackage{graphicx}
\usepackage{longtable}
\usepackage{float}
\usepackage{wrapfig}
\usepackage{soul}
\usepackage{textcomp}
\usepackage{marvosym}
\usepackage{wasysym}
\usepackage{latexsym}
\usepackage{amssymb}
\usepackage{hyperref}
\tolerance=1000
\newcommand{\Slang}{\texttt{S} }
\newcommand{\R}{\texttt{R} }
\newcommand{\Rfunction}[1]{{\texttt{#1}}}
\newcommand{\Robject}[1]{{\texttt{#1}}}
\newcommand{\Rpackage}[1]{{\mbox{\normalfont\textsf{#1}}}}
\usepackage[overlay,absolute]{textpos}
\definecolor{Red}{rgb}{0.7,0,0}
\definecolor{Blue}{rgb}{0,0,0.8}
\usepackage{hyperref}
\hypersetup{%
pdfusetitle,
bookmarks = {true},
bookmarksnumbered = {true},
bookmarksopen = {true},
bookmarksopenlevel = 2,
unicode = {true},
breaklinks = {false},
hyperindex = {true},
colorlinks = {true},
linktocpage = {true},
plainpages = {false},
linkcolor = {Blue},
citecolor = {Blue},
urlcolor = {Red},
pdfstartview = {Fit},
pdfpagemode = {UseOutlines},
pdfview = {XYZ null null null}
}
\AtBeginSection{\begin{frame} \frametitle{Outline} \tableofcontents[currentsection] \end{frame}}
\setbeamersize{text margin left=0.25cm}
\setbeamersize{text margin right=0.25cm}
\setbeamertemplate{navigation symbols}{}
\usepackage{listings}
\providecommand{\alert}[1]{\textbf{#1}}

\title{Starting points for future work}
\author{L Gatto and S J Eglen}
\date{\today}
\hypersetup{
  pdfkeywords={},
  pdfsubject={},
  pdfcreator={Emacs Org-mode version 7.8.11}}

\begin{document}

\maketitle





\makeatletter
\def\DIfF^#1{%
  \mathop{\mathrm{\mathstrut \text{d}}}%
  \nolimits^{#1}\gobblespace}
\makeatother

%% fragwidth will measure the width of the text, and then we use
%% it for the width of the textblock.
\newdimen{\fragwidth}

\newcommand{\mybottomleft}[1]{
\settowidth{\fragwidth}{#1}
\begin{textblock*}{\fragwidth}[0,0](2mm,90mm)  %% {width}(horiz, vert)
  #1
\end{textblock*}
}

\newcommand{\mybottomright}[1]{
\settowidth{\fragwidth}{#1}
\begin{textblock*}{\fragwidth}[1,0](126mm,90mm)  %% {width}(horiz, vert)
  #1
\end{textblock*}
}

\newcommand{\deriv}[3][]{% \deriv[<order>]{<func>}{<var>}
  \ensuremath{\frac{\partial^{#1} {#2}}{\partial {#3}^{#1}}}}







\section{Key unix tools and languages}
\label{sec-1}
\begin{frame}
\frametitle{The shell}
\label{sec-1-1}


The unix \texttt{shell} is an extremely powerful environment that features many 
extremely handy tools, that do simple things, and that can be piped (\texttt{|}) 
together.

\begin{itemize}
\item \texttt{wc}, \texttt{grep}, \texttt{cut}
\item \texttt{tr}, \texttt{sed}, \texttt{awk}
\end{itemize}

\texttt{shell} can also be used for scripting.
\end{frame}
\begin{frame}[fragile]
\frametitle{grep, sed, awk}
\label{sec-1-2}


\begin{itemize}
\item All exploit regular expressions. See ItDT book (later).
\item \texttt{grep}: find matching lines
\item \texttt{sed}: stream-editor.  Incredibly handy for one-liners:
\end{itemize}
\href{http://sed.sourceforge.net/sed1line.txt}{http://sed.sourceforge.net/sed1line.txt}

\begin{verbatim}
 sed 's/foo/bar/g'            # replaces ALL instances in line
 # print section of file between two regular expressions
 sed -n '/Iowa/,/Montana/p'             # case sensitive
\end{verbatim}

\begin{itemize}
\item awk: flexible pattern matching/ processing of text
  files.
\end{itemize}
\href{http://www.pement.org/awk/awk1line.txt}{http://www.pement.org/awk/awk1line.txt}
\begin{verbatim}
 # print the sums of the fields of every line
 awk '{s=0; for (i=1; i<=NF; i++) s=s+$i; print s}'
\end{verbatim}
\end{frame}
\begin{frame}[fragile]
\frametitle{diff: where do my files differ?}
\label{sec-1-3}
\begin{columns}
\begin{column}{0.5\textwidth}
\begin{block}{version1.dat}
\label{sec-1-3-1}

\footnotesize

\begin{verbatim}
0.701 -0.764 -0.226 0.796 -0.337
0.249 -1.51 0.876 2.25 -0.879
-0.523 -1.29 0.354 -0.378 -1.39
0.565 1.31 -0.237 -0.844 0.28
2 -0.128 -0.841 1.31 -0.651
-0.565 0.81 -0.116 0.582 -0.0334
1.03 -0.75 1.7 -0.829 2.3
0.797 -0.988 0.667 -0.492 -0.78
0.94 -0.0931 -0.22 -1.29 -1.21
-0.456 -0.0231 0.603 1.43 0.734
0.598 -0.113 0.852 -1.58 -0.165
0.126 -0.0806 0.951 0.49 0.328
\end{verbatim}
\end{block}
\end{column}
\begin{column}{0.5\textwidth}
\begin{block}{version2.dat}
\label{sec-1-3-2}

\footnotesize

\begin{verbatim}
0.701 -0.764 -0.226 0.796 -0.337
0.249 -1.51 0.876 2.25 -0.879
-0.523 -1.29 0.354 -0.378 -1.39
0.565 1.31 -0.235 -0.844 0.28
2 -0.128 -0.841 1.31 -0.651
-0.565 0.81 -0.116 0.582 -0.0334
1.03 -0.75 1.7 -0.829 2.3
0.797 -0.988 0.667 -0.492 -0.78
0.94 -0.0932 -0.22 -1.29 -1.21
-0.456 -0.0231 0.603 1.43 0.734
0.598 -0.113 0.852 -1.58 -0.165
0.126 -0.0806 0.951 0.49 0.328
\end{verbatim}

   \note{ \url{~/txt/computing/diff_talk/diff_talk_notes.txt} }
\end{block}
\end{column}
\end{columns}
\end{frame}
\begin{frame}[fragile]
\frametitle{diff and patch}
\label{sec-1-4}


\begin{itemize}
\item \emph{diff} shows the differences between version1 and version 2.
\end{itemize}

\begin{verbatim}
 diff nextsteps/version1.dat  nextsteps/version2.dat
\end{verbatim}

\begin{itemize}
\item \emph{patch}:  new file = old file + diff
\end{itemize}


\begin{itemize}
\item \emph{patches} are efficient ways of sending updates.  Useful for syncing
  and version control.
\end{itemize}

\begin{verbatim}
 diff version1.dat version2.dat > p
 patch version1.dat p
 diff version1.dat version2.dat
\end{verbatim}
\end{frame}
\begin{frame}
\frametitle{Perl: Practical Extraction and Report Language}
\label{sec-1-5}


\begin{itemize}
\item Most unix tools (used to be) limited by length of lines.  Perl
  removed those restrictions, combining features of awk, sh and C.
\item `duct tape' programming language.
\item Useful in computational biology.  See \href{http://www.bioperl.org}{http://www.bioperl.org}
\item Excellent Ensembl API, \href{http://www.ensembl.org/info/data/api.html}{http://www.ensembl.org/info/data/api.html}
\item G. Valiente. Combinatorial Pattern Matching Algorithms in Computational Biology using Perl and \R. Taylor \& Francis/CRC Press (2009).
\item Verdict: yucky, but probably [essential | good to now].
\item Bidirectional \R/Perl interfaces \href{http://www.omegahat.org/RSPerl/}{http://www.omegahat.org/RSPerl/}
\end{itemize}
\end{frame}
\begin{frame}
\frametitle{\R can also \texttt{regexp}}
\label{sec-1-6}

\begin{itemize}
\item \texttt{grep}, \texttt{sub}, \texttt{gsub}, \texttt{strsplit}, \texttt{nchar}, \texttt{substr}, \ldots{}
\item also \Rpackage{stringr} package
\end{itemize}

and for sequence data storing and manipulation 

\begin{itemize}
\item \Rpackage{Biostrings} package
\end{itemize}
\end{frame}
\begin{frame}[fragile]
\frametitle{Python}
\label{sec-1-7}


\begin{itemize}
\item Modern programming language; less compact than perl:
\end{itemize}

\footnotesize

\begin{verbatim}
while (<>) {             | import sys
    print if /perl/i;    | for line in sys.stdin.readlines():
}                        |     if line.lower().find("perl") > -1:
                         |         print line,
http://www.sabren.net/articles/againstperl.php3
\end{verbatim}
\normalsize

\begin{itemize}
\item Clean syntax
\item Properly object-oriented.
\item Not as much support in computational biology (yet).  See
  \href{http://www.biopython.org}{http://www.biopython.org}
\item Verdict: More general programming language than \R; lacking (perhaps?)
  in core numerics and graphics -- see NumPy and RPy(2).
\item Bidirectional \R/Python interface \href{http://www.omegahat.org/RSPython/}{http://www.omegahat.org/RSPython/}
\end{itemize}
\end{frame}
\section{\texttt{C} and \texttt{C++} from \R}
\label{sec-2}
\begin{frame}
\frametitle{C}
\label{sec-2-1}


\begin{itemize}
\item Low-level programming language
\item Very fast, but takes a long time to write code.
\item You have to worry about memory allocation yourself.
\item All variables have predefined type.
\item Critical for numerical-intensive work.  (FORTRAN less-popular.)
\end{itemize}
\end{frame}
\begin{frame}
\frametitle{C from \R}
\label{sec-2-2}


\begin{itemize}
\item \R has build-in \texttt{C} interfaces
\begin{itemize}
\item Better know how to program in \texttt{C}.
\item Documentation is not always easy to follow: R-Ext, R Internals as well as \R and other package's code.
\end{itemize}
\item \texttt{.C}  Arguments and return values must be \textit{primitive} (vectors of doubles or integers)
\item \texttt{.Call} Accepts \R data structures as arguments and return values (\texttt{SEXP} and friends) (no type checking is done though).
\item Memory management: memory allocated for \R objects is garbage collected. Thus \R objects in \texttt{C} code, you must be explicitely \texttt{PROTECT}ed to 
      avoid being \texttt{gc()}ed, and subsequently \texttt{UNPROTECT}ed.
\end{itemize}
\end{frame}
\begin{frame}[fragile]
\frametitle{Using \texttt{.Call}}
\label{sec-2-3}

\tiny

\begin{verbatim}
#include <R.h> 
#include <Rdefines.h>

SEXP gccount(SEXP inseq) {
  int i, l;
  SEXP ans, dnaseq;    
  PROTECT(dnaseq = STRING_ELT(inseq, 0)); 
  l = LENGTH(dnaseq); 
  printf("length %d\n",l);
  PROTECT(ans = NEW_NUMERIC(4));

  for (i = 0; i < 4; i++) 
    REAL(ans)[i] = 0;

  for (i = 0; i < l; i++) {
    char p = CHAR(dnaseq)[i];
    if (p=='A') 
      REAL(ans)[0]++;
    else if (p=='C') 
      REAL(ans)[1]++;
    else if (p=='G') 
      REAL(ans)[2]++;
    else if (p=='T') 
      REAL(ans)[3]++;
    else 
      error("Wrong alphabet");
  }
  UNPROTECT(2);
  return(ans);
}
\end{verbatim}
\end{frame}
\begin{frame}[fragile]
\frametitle{Using our \texttt{C} code}
\label{sec-2-4}

\begin{itemize}
\item Create a shared library: \texttt{R CMD SHLIB gccount.c}
\item Load the shared object: \Rfunction{dyn.load("gccount.so")}
\item Create an \R function that uses it: \Rfunction{gccount <- function(inseq) .Call("gccount",inseq)}
\item Use the \texttt{C} code: \Rfunction{gccount("GACAGCATCA")}
\end{itemize}


\begin{verbatim}
s <- "GACTACGA"
gccount
gccount(s)
table(strsplit(s, ""))
system.time(replicate(10000, gccount(s)))
system.time(replicate(10000, table(strsplit(s, ""))))
\end{verbatim}
\end{frame}
\begin{frame}[fragile]
\frametitle{\Rpackage{Rcpp} for \texttt{C++}}
\label{sec-2-5}

\begin{itemize}
\item \Rpackage{Rcpp} is a great package for writing both \texttt{C} and \texttt{C++} code:
\item It comes with loads of documentation and examples.
\item No need to worry about garbage collection.
\item All basic \R types are implemented as \texttt{C++} classes.
\item Easy to interface \texttt{C++} classes (via \texttt{modules})
\item With package \Rpackage{inline} code can be easily compiled in \R.
\end{itemize}
\small

\begin{verbatim}
library(Rcpp)
library(inline)
cppCode <- '
    Rcpp::NumericVector cx(x);
    Rcpp::NumericVector ret(1);
    ret[0] = cx[0] * cx[0];
    return(ret);
    '
squareOne <- cxxfunction(signature(x="numeric"), 
                      plugin="Rcpp", body=cppCode)
squareOne(10)
\end{verbatim}
\end{frame}
\section{Not only local}
\label{sec-3}
\begin{frame}
\frametitle{Syncing your files}
\label{sec-3-1}


\begin{itemize}
\item How do you keep two directories in synchrony, e.g. your home
  directory on laptop and desktop?
\item \texttt{sftp}, \texttt{ssh}, \texttt{rsync}
\item \texttt{Unison} gets Stephen's vote since 2003 -- \href{http://www.damtp.cam.ac.uk/internal/computing/unison/}{http://www.damtp.cam.ac.uk/internal/computing/unison/}
\item Modern services like Dropbox are useful and build upon these unix
  tools.
\end{itemize}
\end{frame}
\begin{frame}
\frametitle{Version control / revision control system (RCS)}
\label{sec-3-2}


\begin{itemize}
\item How to keep backup copies over time?
\item Just copy files, e.g. \emph{mycode.jan1.R}, \emph{mycode.jan2.R}, \ldots{}
\item Leads to many large copies, with no trace of what you did over time.
\item more principled way is to use version control: every time you make
  significant changes, you \emph{commit} a new version with a succint log
  file saying what you changed.
\item RCS: going since 1982\ldots{} old and simple but stable.  Typically
  single-user.
\end{itemize}
\url{http://www.cl.cam.ac.uk/~mgk25/rcsintro.html}

\begin{itemize}
\item More modern approaches: \emph{cvs}, \emph{svn}, \emph{git}, \ldots{}
\end{itemize}
\end{frame}
\begin{frame}
\frametitle{For packages, analysis projects, papers and slides}
\label{sec-3-3}

\begin{itemize}
\item Github, google code, bitbucket, \ldots{}
\item R-forge: svn and build system
\end{itemize}
\end{frame}
\section{Handling large files / databases}
\label{sec-4}
\begin{frame}
\frametitle{Handling large data files.}
\label{sec-4-1}

\begin{itemize}
\item Computational Biology requires access to large data files.
\item Reading them all into memory is difficult, when files are very large
     (> 1 Gb).
\item Some approaches:
\begin{enumerate}
\item Compress files.
\item Selectively use scan or connections.
\item Use a database.
\end{enumerate}
\end{itemize}
\end{frame}
\begin{frame}[fragile]
\frametitle{1. Compress files.}
\label{sec-4-2}


\begin{itemize}
\item This produces typically x2 compression:
\end{itemize}

\begin{verbatim}
 Rscript -e 'write(rnorm(99999), file="largefile.dat")'
 ls -lh largefile.dat
 gzip largefile.dat
 ls -lh largefile.dat.gz
 gunzip largefile.dat
\end{verbatim}

\begin{itemize}
\item \R can read in compressed files natively.
\end{itemize}

\begin{verbatim}
 x <- scan('largefile.dat.gz')
\end{verbatim}

\begin{itemize}
\item Other compression options also recognised: xz, bzip2
\end{itemize}
\end{frame}
\begin{frame}[fragile]
\frametitle{2. Scan and Connections.}
\label{sec-4-3}


\begin{itemize}
\item scan() is very flexible; e.g. read just 2nd column:
\end{itemize}
\footnotesize

\begin{verbatim}
scan(file = "", what = double(0), nmax = -1, n = -1, sep = "",
     quote = if(identical(sep, "\n")) "" else "'\"", dec = ".",
     skip = 0, nlines = 0, na.strings = "NA",
     flush = FALSE, fill = FALSE, strip.white = FALSE,
     quiet = FALSE, blank.lines.skip = TRUE, multi.line = TRUE,
     comment.char = "", allowEscapes = FALSE,
     fileEncoding = "", encoding = "unknown")

x <- scan(file, what=list(NULL,"",NULL), skip=2, sep='\t')
\end{verbatim}
\normalsize

\begin{itemize}
\item connections allow you to maintain state between accesses to a file.
\end{itemize}

\footnotesize

\begin{verbatim}
con <- file("version1.dat", "r")
while (length(dat <- scan(con,n = 5,quiet = TRUE)) > 0) {
  print(mean(dat))
}
close(con)
\end{verbatim}
\end{frame}
\begin{frame}
\frametitle{3. Relational databases}
\label{sec-4-4}


\begin{itemize}
\item Relational database: data stored in tables, very similar in nature
  to \R's data.frames.
\item Databases allow for multiple-accesses, locks for restricted changes,
  very scalable.
\item Many databases available: Oracle, Postgres, Access, MySQL.
\item SQL -- Structured Query Language: language to interrogate databses.
\end{itemize}
\end{frame}
\begin{frame}
\frametitle{What is SQLite?}
\label{sec-4-5}


\begin{itemize}
\item Most databases run on remote server; SQLite is embedded into your
  program.
\item Embedding the database simplifies setup of server, but means your
  databases are not shared in the same way that others are.  (You have
  to share the .sql files.)
\item Incredibly small (1/4 Mb) and useful.  Widely used (e.g. mac, iOS,
  Firefox, Android).  Not as fast as e.g. Oracle.
\item You compile your SQLite within your program.
\item All handled with you by \R, care of \emph{RSQLite} package.
  (e.g. Bioconductor uses it for data files.)
\end{itemize}
\end{frame}
\begin{frame}[fragile]
\frametitle{Using databases in \R, a simple session (Gentleman, p239)}
\label{sec-4-6}


\begin{itemize}
\item package \emph{DBI} interfaces to all database platforms.
\end{itemize}


\begin{verbatim}
library(RSQLite)
m = dbDriver("SQLite")

## Create a new database from an R data frame.
con = dbConnect(m, dbname = "arrest.db")
data(USArrests)
dbWriteTable(con, "USArrests", USArrests, overwrite=TRUE)
dbListTables(con)

## Later, query the database.
rs = dbSendQuery(con, "select * from USArrests")
d1 = fetch(rs, n=5)      ## get first five
print(d1)
d1 = fetch(rs, n=-1)
dbDisconnect(con)
\end{verbatim}
\end{frame}
\begin{frame}[fragile]
\frametitle{Other uses for sqlite}
\label{sec-4-7}


\begin{itemize}
\item \texttt{sqldf} Performs SQL selects on \R data frames.
\item supports SQLite backend database (by default), the H2 java db and PostgreSQL and MySQL.
\item avoid read.csv entirely  \href{http://code.google.com/p/sqldf/}{http://code.google.com/p/sqldf/}
\end{itemize}

\begin{quote}
``See ?read.csv.sql in sqldf.  It uses RSQLite and SQLite to read the
file into an sqlite database (which it sets up for you) completely
bypassing \R and from there grabs it into \R removing the database it
created at the end.'' (G. Grothendieck, r-help mailing list).
\end{quote}

\begin{itemize}
\item Good book: \verb~^((HT|X)M|SQ)L|R$~

  Introduction to Data Technologies.
\end{itemize}

\url{http://www.stat.auckland.ac.nz/~paul/ItDT/}
\end{frame}
\begin{frame}
\frametitle{\texttt{ff}: back to the future?}
\label{sec-4-8}


\begin{itemize}
\item \emph{ff} package stores objects on disk, but looks like they are in
   memory.
\item ``back to the future'': S used to store objects in disk.
\item Sorting a single column of 81e6 entries.  Time-taken in seconds.
\end{itemize}

Oct 2010 results from.
\href{http://tolstoy.newcastle.edu.au/R/packages/10/0697.html}{http://tolstoy.newcastle.edu.au/R/packages/10/0697.html}



\begin{center}
\begin{tabular}{lrrllllll}
\hline
      &  ruinteger  &  rinteger  &  rusingle  &  rsingle  &  rudouble  &  rdouble  &  rfactor  &  rchar  \\
\hline
 ram  &       5.58  &      3.23  &  NA        &  NA       &  NA        &  NA       &  0.49     &  NA     \\
 ff   &      10.70  &      8.54  &  51.35     &  28.98    &  70.20     &  44.13    &  7.91     &  NA     \\
 R    &        OOM  &       OOM  &  OOM       &  OOM      &  OOM       &  OOM      &  OOM      &  OOM    \\
 SAS  &      61.45  &     44.94  &  NA        &  NA       &  63.14     &  46.56    &  NA       &  OOD    \\
\hline
\end{tabular}
\end{center}



(ram=in-memory, optimized for speed, not ram; ff=on disk).

\note{see text in nextsteps/ff-oct2010.txt}
\end{frame}
\begin{frame}
\frametitle{Other}
\label{sec-4-9}


\begin{itemize}
\item The \texttt{bigmemory} package by Kane and Emerson permits storing large objects such as matrices in memory (as well as via files) and uses external pointer objects to refer to them.
\item netCDF data files: \texttt{ncdf} and \texttt{RNetCDF} packages.
\item hdf5 format: \texttt{rhdf5} package
\item \texttt{XML} package to parse xml data
\item \ldots{}
\end{itemize}
\end{frame}
\section{Parallel processing}
\label{sec-5}
\begin{frame}
\frametitle{Introduction}
\label{sec-5-1}

\begin{itemize}
\item Applicable when repeating \textit{independent} computations a certain number of times; results just need to be combined after parallel executions are done.
\item A cluster of nodes: generate multiple workers listening to the master; these workers are new processes that can run on the current machine or a similar one with an identical R installation. Should work on all \R plateforms (as in package \Rpackage{snow}).
\item The \R process is \textit{forked} to create new \R\~{}processes by taking a complete copy of the masters process, including workspace (pioneered by package \Rpackage{multicore}). Does not work on Windows.
\item Grid computing.
\end{itemize}
\end{frame}
\begin{frame}[fragile]
\frametitle{Packages}
\label{sec-5-2}

\begin{itemize}
\item Package \Rpackage{parallel}, first included in \R 2.14.0 builds on CRAN packages \Rpackage{multicore} and \Rpackage{snow}.
\end{itemize}
\begin{verbatim}
 mclapply(X, FUN, ...) (adapted from multicore).
 parLapply(cl, X, FUN, ...) (adapted from snow ).
\end{verbatim}

\begin{itemize}
\item Package \Rpackage{foreach}, introducing a new looping construct supporting parallel execution. Natural choice to parallelise a \texttt{for} loop.
\end{itemize}
\begin{verbatim}
 library(doMC)
 library(foreach)
 registerDoMC(2)
 foreach(i = ll) %dopar% f(i)
 foreach(i = ll) %do% f(i) ## serial version
 library(plyr)
 llply(ll, f, .parallel=TRUE)
\end{verbatim}
\end{frame}
\begin{frame}
\frametitle{High performance computing}
\label{sec-5-3}


\begin{itemize}
\item Find information about managing and chunking big data:
\begin{itemize}
\item High performance computing CRAN task view
\item \href{http://cran.r-project.org/web/views/HighPerformanceComputing.html}{http://cran.r-project.org/web/views/HighPerformanceComputing.html}
\end{itemize}
\end{itemize}
\end{frame}
\section{A few more words about about \R}
\label{sec-6}
\begin{frame}[fragile]
\frametitle{Pass by \ldots{}}
\label{sec-6-1}


\begin{itemize}
\item \textbf{value} is the default in \R
\item \textbf{reference} using S4 ReferenceClasses (OO)
\item can emulate pass by ref using an \texttt{environment}
\end{itemize}


\begin{verbatim}
e <- new.env()
e$x <- 1
f <- function(myenv) myenv$x <- 2
f(e)
e$x
\end{verbatim}
\end{frame}
\begin{frame}[fragile]
\frametitle{Profiling}
\label{sec-6-2}



\begin{verbatim}
m <- matrix(rnorm(1e6), ncol=100)
Rprof("rprof")                                
res <- apply(m,1,mean,trim=.3)                
Rprof(NULL)
summaryRprof("rprof")
\end{verbatim}
\end{frame}
\begin{frame}[fragile]
\frametitle{Benchmarking}
\label{sec-6-3}



\begin{verbatim}
m <- matrix(rnorm(1e6), ncol=100)
f1 <- function(x, t = 0.3) {
  xx <- 0
  for (i in 1:nrow(x)) {
    xx <- c(xx, sum(m[i, ]))
  }
  mean(xx, trim = t)
}
f2 <- function(x, t = 0.3) mean(rowSums(x), trim = t)

library(rbenchmark)
benchmark(f1(m), f2(m),
          columns=c("test", "replications", 
            "elapsed", "relative"),
          order = "relative", replications = 10)
\end{verbatim}
\end{frame}
\section{Differential equations and phase plane analysis}
\label{sec-7}
\begin{frame}
\frametitle{Lotka–Volterra equations}
\label{sec-7-1}


Describe simple models of populations dynamics of species competing for some common resource. 
When two species are not interacting, their population evolve according to the logistic 
equations and the rate of reproduction is proportional to both the existing population 
and the amount of available resources

\begin{align*}
 \deriv{x}{t} &= r_{1} x ~ (1 - \frac{x}{k_{1}} )\\
 \deriv{y}{t} &= r_{2} y ~ (1 - \frac{y}{k_{2}} )
\end{align*}

where the constant $r_{i}$ defines the growth rate and $k_{i}$ is the carrying capacity of the environment.
\end{frame}
\begin{frame}
\frametitle{Competitive Lotka–Volterra equations}
\label{sec-7-2}


When competing for the same resource, the animals have a negative influence on their competitors growth.

\begin{align*}
 \deriv{x}{t} &= r_{1} x ~ (1 - \frac{x}{k_{1}} ) - axy\\
 \deriv{y}{t} &= r_{2} y ~ (1 - \frac{y}{k_{2}} ) - bxy
\end{align*}
\end{frame}
\begin{frame}
\frametitle{Rabbits vs sheep  (Strogatz, p155)}
\label{sec-7-3}


Here is an example with $r_{1} = 3$, $k_{1} = 3$, $a = 2$, $r_{2} = 2$, $k_{2} = 2$, $b = 1$, which simplifies to

\begin{align*}
 \deriv{r}{t} &= r( 3 - r - 2s)\\
 \deriv{s}{t} &= s( 2 - r - s)
\end{align*}
\end{frame}
\begin{frame}[fragile]
\frametitle{Computing a trajectory over time}
\label{sec-7-4}


i.e. use numerical integration, with $r_0 = 1$ and $s_0 = 1.2$


\begin{verbatim}
library(deSolve)
Sheep <- function(t, y, parms) {
  r=y[1]; s=y[2]
  drdt = r * (3 - r - (2*s))
  dsdt = s * (2 - r - s)
  list(c(drdt, dsdt))
}

x0 <- c(1, 1.2)
times <- seq(0, 30, by=0.2)
parms <- 0
out <- rk4(x0, times, Sheep, parms)
head(out)
\end{verbatim}
\end{frame}
\begin{frame}
\frametitle{Plotting population growth}
\label{sec-7-5}


\includegraphics[width=.9\linewidth]{./de/sheep_run_01.pdf}
\end{frame}
\begin{frame}
\frametitle{Phase plane analysis}
\label{sec-7-6}


\includegraphics[width=.9\linewidth]{./de/sheep_phase.pdf}
\end{frame}
\begin{frame}
\frametitle{Starting points}
\label{sec-7-7}


\begin{itemize}
\item \texttt{deSolve} package
\item phase planes and nullclines (\texttt{DMBpplane.r} from DMB site, modified
     from Daniel Kaplan)
\item \texttt{integrate()}  -- quadrature
\item \texttt{D()} -- symbolic differentiation
\item \texttt{optimize()} (1d) and \texttt{optim()} (n-d)
\item Steven Strogatz.  Nonlinear dynamics and chaos.
\item NR: William Press et al.  Numerical Recipes in C/C++
\item More slides about DE and phase plane -- \url{de.pdf}
\end{itemize}
\end{frame}
\section{Conclusions}
\label{sec-8}
\begin{frame}
\frametitle{Conclusions}
\label{sec-8-1}


\begin{itemize}
\item Looking for packages
\begin{itemize}
\item CRAN Task Views \href{http://cran.r-project.org/web/views/}{http://cran.r-project.org/web/views/}
\item Bioconductor biocViews \href{http://bioconductor.org/packages/release/BiocViews.html}{http://bioconductor.org/packages/release/BiocViews.html}
\end{itemize}
\end{itemize}


\begin{itemize}
\item Reproducibility is crucial
\item Have several tools at hand
\begin{itemize}
\item editor, programming languages, shell, \ldots{}
\end{itemize}
\item Practice to keep learning
\item Have fun! \smiley{}
\end{itemize}
\end{frame}

\end{document}
